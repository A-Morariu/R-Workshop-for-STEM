\documentclass[]{article}
\usepackage{lmodern}
\usepackage{amssymb,amsmath}
\usepackage{ifxetex,ifluatex}
\usepackage{fixltx2e} % provides \textsubscript
\ifnum 0\ifxetex 1\fi\ifluatex 1\fi=0 % if pdftex
  \usepackage[T1]{fontenc}
  \usepackage[utf8]{inputenc}
\else % if luatex or xelatex
  \ifxetex
    \usepackage{mathspec}
  \else
    \usepackage{fontspec}
  \fi
  \defaultfontfeatures{Ligatures=TeX,Scale=MatchLowercase}
\fi
% use upquote if available, for straight quotes in verbatim environments
\IfFileExists{upquote.sty}{\usepackage{upquote}}{}
% use microtype if available
\IfFileExists{microtype.sty}{%
\usepackage{microtype}
\UseMicrotypeSet[protrusion]{basicmath} % disable protrusion for tt fonts
}{}
\usepackage[margin=1in]{geometry}
\usepackage{hyperref}
\hypersetup{unicode=true,
            pdftitle={STEM Fellowship R Workshop},
            pdfauthor={Alin Morariu},
            pdfborder={0 0 0},
            breaklinks=true}
\urlstyle{same}  % don't use monospace font for urls
\usepackage{color}
\usepackage{fancyvrb}
\newcommand{\VerbBar}{|}
\newcommand{\VERB}{\Verb[commandchars=\\\{\}]}
\DefineVerbatimEnvironment{Highlighting}{Verbatim}{commandchars=\\\{\}}
% Add ',fontsize=\small' for more characters per line
\usepackage{framed}
\definecolor{shadecolor}{RGB}{248,248,248}
\newenvironment{Shaded}{\begin{snugshade}}{\end{snugshade}}
\newcommand{\AlertTok}[1]{\textcolor[rgb]{0.94,0.16,0.16}{#1}}
\newcommand{\AnnotationTok}[1]{\textcolor[rgb]{0.56,0.35,0.01}{\textbf{\textit{#1}}}}
\newcommand{\AttributeTok}[1]{\textcolor[rgb]{0.77,0.63,0.00}{#1}}
\newcommand{\BaseNTok}[1]{\textcolor[rgb]{0.00,0.00,0.81}{#1}}
\newcommand{\BuiltInTok}[1]{#1}
\newcommand{\CharTok}[1]{\textcolor[rgb]{0.31,0.60,0.02}{#1}}
\newcommand{\CommentTok}[1]{\textcolor[rgb]{0.56,0.35,0.01}{\textit{#1}}}
\newcommand{\CommentVarTok}[1]{\textcolor[rgb]{0.56,0.35,0.01}{\textbf{\textit{#1}}}}
\newcommand{\ConstantTok}[1]{\textcolor[rgb]{0.00,0.00,0.00}{#1}}
\newcommand{\ControlFlowTok}[1]{\textcolor[rgb]{0.13,0.29,0.53}{\textbf{#1}}}
\newcommand{\DataTypeTok}[1]{\textcolor[rgb]{0.13,0.29,0.53}{#1}}
\newcommand{\DecValTok}[1]{\textcolor[rgb]{0.00,0.00,0.81}{#1}}
\newcommand{\DocumentationTok}[1]{\textcolor[rgb]{0.56,0.35,0.01}{\textbf{\textit{#1}}}}
\newcommand{\ErrorTok}[1]{\textcolor[rgb]{0.64,0.00,0.00}{\textbf{#1}}}
\newcommand{\ExtensionTok}[1]{#1}
\newcommand{\FloatTok}[1]{\textcolor[rgb]{0.00,0.00,0.81}{#1}}
\newcommand{\FunctionTok}[1]{\textcolor[rgb]{0.00,0.00,0.00}{#1}}
\newcommand{\ImportTok}[1]{#1}
\newcommand{\InformationTok}[1]{\textcolor[rgb]{0.56,0.35,0.01}{\textbf{\textit{#1}}}}
\newcommand{\KeywordTok}[1]{\textcolor[rgb]{0.13,0.29,0.53}{\textbf{#1}}}
\newcommand{\NormalTok}[1]{#1}
\newcommand{\OperatorTok}[1]{\textcolor[rgb]{0.81,0.36,0.00}{\textbf{#1}}}
\newcommand{\OtherTok}[1]{\textcolor[rgb]{0.56,0.35,0.01}{#1}}
\newcommand{\PreprocessorTok}[1]{\textcolor[rgb]{0.56,0.35,0.01}{\textit{#1}}}
\newcommand{\RegionMarkerTok}[1]{#1}
\newcommand{\SpecialCharTok}[1]{\textcolor[rgb]{0.00,0.00,0.00}{#1}}
\newcommand{\SpecialStringTok}[1]{\textcolor[rgb]{0.31,0.60,0.02}{#1}}
\newcommand{\StringTok}[1]{\textcolor[rgb]{0.31,0.60,0.02}{#1}}
\newcommand{\VariableTok}[1]{\textcolor[rgb]{0.00,0.00,0.00}{#1}}
\newcommand{\VerbatimStringTok}[1]{\textcolor[rgb]{0.31,0.60,0.02}{#1}}
\newcommand{\WarningTok}[1]{\textcolor[rgb]{0.56,0.35,0.01}{\textbf{\textit{#1}}}}
\usepackage{graphicx,grffile}
\makeatletter
\def\maxwidth{\ifdim\Gin@nat@width>\linewidth\linewidth\else\Gin@nat@width\fi}
\def\maxheight{\ifdim\Gin@nat@height>\textheight\textheight\else\Gin@nat@height\fi}
\makeatother
% Scale images if necessary, so that they will not overflow the page
% margins by default, and it is still possible to overwrite the defaults
% using explicit options in \includegraphics[width, height, ...]{}
\setkeys{Gin}{width=\maxwidth,height=\maxheight,keepaspectratio}
\IfFileExists{parskip.sty}{%
\usepackage{parskip}
}{% else
\setlength{\parindent}{0pt}
\setlength{\parskip}{6pt plus 2pt minus 1pt}
}
\setlength{\emergencystretch}{3em}  % prevent overfull lines
\providecommand{\tightlist}{%
  \setlength{\itemsep}{0pt}\setlength{\parskip}{0pt}}
\setcounter{secnumdepth}{0}
% Redefines (sub)paragraphs to behave more like sections
\ifx\paragraph\undefined\else
\let\oldparagraph\paragraph
\renewcommand{\paragraph}[1]{\oldparagraph{#1}\mbox{}}
\fi
\ifx\subparagraph\undefined\else
\let\oldsubparagraph\subparagraph
\renewcommand{\subparagraph}[1]{\oldsubparagraph{#1}\mbox{}}
\fi

%%% Use protect on footnotes to avoid problems with footnotes in titles
\let\rmarkdownfootnote\footnote%
\def\footnote{\protect\rmarkdownfootnote}

%%% Change title format to be more compact
\usepackage{titling}

% Create subtitle command for use in maketitle
\providecommand{\subtitle}[1]{
  \posttitle{
    \begin{center}\large#1\end{center}
    }
}

\setlength{\droptitle}{-2em}

  \title{STEM Fellowship R Workshop}
    \pretitle{\vspace{\droptitle}\centering\huge}
  \posttitle{\par}
    \author{Alin Morariu}
    \preauthor{\centering\large\emph}
  \postauthor{\par}
      \predate{\centering\large\emph}
  \postdate{\par}
    \date{30/10/2019}


\begin{document}
\maketitle

\hypertarget{welcome}{%
\section{Welcome!}\label{welcome}}

This is a 2 part workshop focusing on key coding skills in R that will
be useful in a research setting. Topics covered will include

\begin{itemize}
\tightlist
\item
  R objects
\item
  Basic operations
\item
  Introduction to knitr
\item
  Introduction to the tidyverse
\item
  Data wrangling (via dplyr)
\item
  Data visualization (via ggplot2)
\item
  Introduction to hypothesis testing
\item
  Linear models (if time permits)
\item
  Export our results to a PDF via an RMarkdown file
\end{itemize}

\hypertarget{what-is-r}{%
\subsubsection{What is R?}\label{what-is-r}}

R is a statistical program language that gives you a means of computing
key values and plots to analyze your data. We will be using an IDE
overlayed onto R called R Studio which provies a clean, intuitive
environment for you to work in. R Studio has 4 main panes - Source file,
Console, Environment, and Files.

\hypertarget{r-objects}{%
\subsubsection{R objects}\label{r-objects}}

There are many ways to store data/information in R. Here are a few of
the key objects we will come across in our work.

\begin{Shaded}
\begin{Highlighting}[]
\CommentTok{# storing a number }
\NormalTok{x0 <-}\StringTok{ }\DecValTok{42} 
\KeywordTok{class}\NormalTok{(x0)}
\end{Highlighting}
\end{Shaded}

\begin{verbatim}
## [1] "numeric"
\end{verbatim}

Note how x0 now appears in your environment and points to the value 42
stored in memory.

\begin{Shaded}
\begin{Highlighting}[]
\CommentTok{# can store multiple values in the same memory slot }
\NormalTok{v0 <-}\StringTok{ }\KeywordTok{c}\NormalTok{(}\DecValTok{41}\NormalTok{, x0, }\DecValTok{43}\NormalTok{)}
\NormalTok{v0}
\end{Highlighting}
\end{Shaded}

\begin{verbatim}
## [1] 41 42 43
\end{verbatim}

\begin{Shaded}
\begin{Highlighting}[]
\CommentTok{# access specific values of a vector }
\NormalTok{v0[}\DecValTok{2}\NormalTok{]}
\end{Highlighting}
\end{Shaded}

\begin{verbatim}
## [1] 42
\end{verbatim}

\begin{Shaded}
\begin{Highlighting}[]
\NormalTok{v0[}\DecValTok{2}\OperatorTok{:}\DecValTok{3}\NormalTok{]}
\end{Highlighting}
\end{Shaded}

\begin{verbatim}
## [1] 42 43
\end{verbatim}

\begin{Shaded}
\begin{Highlighting}[]
\CommentTok{# can store multiple vectors in a data frame }
\NormalTok{df0 <-}\StringTok{ }\KeywordTok{data.frame}\NormalTok{(v0, }\DecValTok{2}\OperatorTok{*}\NormalTok{v0, v0}\OperatorTok{/}\DecValTok{2}\NormalTok{)}
\NormalTok{df0}
\end{Highlighting}
\end{Shaded}

\begin{verbatim}
##   v0 X2...v0 v0.2
## 1 41      82 20.5
## 2 42      84 21.0
## 3 43      86 21.5
\end{verbatim}

\hypertarget{r-operations}{%
\subsubsection{R operations}\label{r-operations}}

This makes for a nice transition into some basic operations you can do
with R. It is helpful to think of R as an incredibly powerful
calculator.

\begin{Shaded}
\begin{Highlighting}[]
\CommentTok{# addition and subtraction }
\NormalTok{x0 }\OperatorTok{+}\StringTok{ }\NormalTok{x0}
\end{Highlighting}
\end{Shaded}

\begin{verbatim}
## [1] 84
\end{verbatim}

\begin{Shaded}
\begin{Highlighting}[]
\NormalTok{x0 }\OperatorTok{-}\StringTok{ }\DecValTok{42}
\end{Highlighting}
\end{Shaded}

\begin{verbatim}
## [1] 0
\end{verbatim}

\begin{Shaded}
\begin{Highlighting}[]
\CommentTok{# multiplication and subtraction }
\DecValTok{2}\OperatorTok{*}\NormalTok{x0}
\end{Highlighting}
\end{Shaded}

\begin{verbatim}
## [1] 84
\end{verbatim}

\begin{Shaded}
\begin{Highlighting}[]
\NormalTok{x0}\OperatorTok{*}\DecValTok{3} 
\end{Highlighting}
\end{Shaded}

\begin{verbatim}
## [1] 126
\end{verbatim}

\begin{Shaded}
\begin{Highlighting}[]
\DecValTok{2}\OperatorTok{*}\NormalTok{v0}
\end{Highlighting}
\end{Shaded}

\begin{verbatim}
## [1] 82 84 86
\end{verbatim}

\begin{Shaded}
\begin{Highlighting}[]
\DecValTok{2}\OperatorTok{*}\NormalTok{df0 }\CommentTok{# note at how the scalar is applied }
\end{Highlighting}
\end{Shaded}

\begin{verbatim}
##   v0 X2...v0 v0.2
## 1 82     164   41
## 2 84     168   42
## 3 86     172   43
\end{verbatim}

Some times we need to multiply large matricies in order to compute
certain values quickly. Fortunately, R provides functions dedicated to
doing just this.

\begin{Shaded}
\begin{Highlighting}[]
\CommentTok{# define a 3x3 matrix}
\NormalTok{mt0 <-}\StringTok{ }\KeywordTok{as.matrix}\NormalTok{(df0)}
\NormalTok{mt0}
\end{Highlighting}
\end{Shaded}

\begin{verbatim}
##      v0 X2...v0 v0.2
## [1,] 41      82 20.5
## [2,] 42      84 21.0
## [3,] 43      86 21.5
\end{verbatim}

\begin{Shaded}
\begin{Highlighting}[]
\CommentTok{# transpose}
\KeywordTok{t}\NormalTok{(mt0)}
\end{Highlighting}
\end{Shaded}

\begin{verbatim}
##         [,1] [,2] [,3]
## v0      41.0   42 43.0
## X2...v0 82.0   84 86.0
## v0.2    20.5   21 21.5
\end{verbatim}

\begin{Shaded}
\begin{Highlighting}[]
\CommentTok{# vector multiplication }

\CommentTok{# element wise }
\NormalTok{v0 }\OperatorTok{*}\StringTok{ }\NormalTok{v0 }
\end{Highlighting}
\end{Shaded}

\begin{verbatim}
## [1] 1681 1764 1849
\end{verbatim}

\begin{Shaded}
\begin{Highlighting}[]
\CommentTok{# dot product }
\NormalTok{v0 }\OperatorTok\StringTok{ }\NormalTok{v0}
\end{Highlighting}
\end{Shaded}

\begin{verbatim}
##      [,1]
## [1,] 5294
\end{verbatim}

\begin{Shaded}
\begin{Highlighting}[]
\CommentTok{# vector product}
\NormalTok{v0 }\OperatorTok\StringTok{ }\KeywordTok{t}\NormalTok{(v0)}
\end{Highlighting}
\end{Shaded}

\begin{verbatim}
##      [,1] [,2] [,3]
## [1,] 1681 1722 1763
## [2,] 1722 1764 1806
## [3,] 1763 1806 1849
\end{verbatim}

\begin{Shaded}
\begin{Highlighting}[]
\CommentTok{# matrix product }
\NormalTok{mt0 }\OperatorTok\StringTok{ }\NormalTok{mt0 }
\end{Highlighting}
\end{Shaded}

\begin{verbatim}
##          v0 X2...v0    v0.2
## [1,] 6006.5   12013 3003.25
## [2,] 6153.0   12306 3076.50
## [3,] 6299.5   12599 3149.75
\end{verbatim}

With that last set of operations, we must be careful to ensure that the
dimensions of the vectors make sense with respect to the products.

\hypertarget{functions}{%
\subsubsection{Functions}\label{functions}}

Sometimes we want to do a defined set of operations and this is where
user functions come in. R contains a set of built in functions which
come in handy, but we can also define our own functions.

\begin{Shaded}
\begin{Highlighting}[]
\CommentTok{# built in function example }
\KeywordTok{sum}\NormalTok{(}\DecValTok{1}\NormalTok{,}\DecValTok{2}\NormalTok{,}\DecValTok{3}\NormalTok{)}
\end{Highlighting}
\end{Shaded}

\begin{verbatim}
## [1] 6
\end{verbatim}

\begin{Shaded}
\begin{Highlighting}[]
\KeywordTok{sum}\NormalTok{(v0)}
\end{Highlighting}
\end{Shaded}

\begin{verbatim}
## [1] 126
\end{verbatim}

\begin{Shaded}
\begin{Highlighting}[]
\NormalTok{my_sum <-}\StringTok{ }\ControlFlowTok{function}\NormalTok{(vector)\{}
  \CommentTok{# this function will sum the first and last numbers in a vector }
\NormalTok{  x <-}\StringTok{ }\NormalTok{vector[}\DecValTok{1}\NormalTok{] }\OperatorTok{+}\StringTok{ }\NormalTok{vector[}\KeywordTok{length}\NormalTok{(vector)]}
  \KeywordTok{return}\NormalTok{(x)}
\NormalTok{\}}

\KeywordTok{my_sum}\NormalTok{(v0) }\CommentTok{# can you think of a case where this function will fail? }
\end{Highlighting}
\end{Shaded}

\begin{verbatim}
## [1] 84
\end{verbatim}

\hypertarget{packages}{%
\subsubsection{Packages}\label{packages}}

Since base R could not possilbe incorporate all functions anyone has
ever made, users will sometimes ``package'' their user defined functions
into an object called a package. Packages are simply scripts of code
which you can import into your work space and take advantage of in order
to simplify your work. The advantage to packages is that all of their
source code is hidden from you, the user, but you get access to a wide
array of very powerful and often highly optimized functions (more on
this later).

\begin{Shaded}
\begin{Highlighting}[]
\CommentTok{# before using a package, you need to install it}
\KeywordTok{install.packages}\NormalTok{(}\StringTok{"tidyverse"}\NormalTok{)}
\KeywordTok{install.packages}\NormalTok{(}\StringTok{"knitr)}
\end{Highlighting}
\end{Shaded}

The first time you ever want to use a package, you must install it so R
has the source code. From this point on, any time you want to use the
package all you need to do is call it so R knows to activate this code
via the library function

\begin{Shaded}
\begin{Highlighting}[]
\KeywordTok{library}\NormalTok{(tidyverse)}
\end{Highlighting}
\end{Shaded}

\begin{verbatim}
## -- Attaching packages ------------------------------------------------------------ tidyverse 1.2.1 --
\end{verbatim}

\begin{verbatim}
## v ggplot2 3.2.1     v purrr   0.3.3
## v tibble  2.1.3     v dplyr   0.8.3
## v tidyr   1.0.0     v stringr 1.4.0
## v readr   1.3.1     v forcats 0.4.0
\end{verbatim}

\begin{verbatim}
## -- Conflicts --------------------------------------------------------------- tidyverse_conflicts() --
## x dplyr::filter() masks stats::filter()
## x dplyr::lag()    masks stats::lag()
\end{verbatim}

\begin{Shaded}
\begin{Highlighting}[]
\KeywordTok{library}\NormalTok{(knitr)}
\end{Highlighting}
\end{Shaded}

Today we will focus on the usage of two key R packages - tidyverse and
knitr.

\hypertarget{tidyverse}{%
\subsubsection{Tidyverse}\label{tidyverse}}

The tidyverse is a conglometration of smaller packages created with the
goal of simplifying many of R's data science capabilities as well as
taking advantage of modern computing power through highly efficient back
end code (you do NOT need to worry about this and this is why having it
in a package is so convenient). Within the tidyverse, we will focus on 2
main sub-packages. Namely, dplyr (a package used for data wrangling and
cleaning), and ggplot2 (a packaged used for data visualization and
creation of plots).

\hypertarget{knitr}{%
\subsubsection{Knitr}\label{knitr}}

Knitr will be particularly useful to us since it will allow us to export
our work from these .Rmd files into usable, and sometimes publishable,
documents. The RMarkdown documents can be exported to HTML, PDF, or Word
documents making them very useful when working on a project. The
alternative is to write all of your code in an R script (a .R file), and
work with the outputted objects individually (for example, you can save
any of your plots as .jpeg or .png files which you can then slot into
your figures in another document). I recommend using the RMarkdown
setting due to it's ability to handle inline code via the code chunks
you've been seeing so far.

\hypertarget{working-with-data}{%
\section{Working with Data}\label{working-with-data}}

Now that we have a basic understanding what R is capable of doing at an
``atomic'' level, let us pull in some data and do a low level analysis.
We will begin by cleaning the data first, then plotting the data.

\hypertarget{cleaning}{%
\subsubsection{Cleaning}\label{cleaning}}

\begin{Shaded}
\begin{Highlighting}[]
\NormalTok{trees <-}\StringTok{ }\NormalTok{datasets}\OperatorTok{::}\NormalTok{trees}

\CommentTok{# preview our data}
\KeywordTok{head}\NormalTok{(trees)}
\end{Highlighting}
\end{Shaded}

\begin{verbatim}
##   Girth Height Volume
## 1   8.3     70   10.3
## 2   8.6     65   10.3
## 3   8.8     63   10.2
## 4  10.5     72   16.4
## 5  10.7     81   18.8
## 6  10.8     83   19.7
\end{verbatim}

\begin{Shaded}
\begin{Highlighting}[]
\CommentTok{# another option}
\KeywordTok{glimpse}\NormalTok{(trees)}
\end{Highlighting}
\end{Shaded}

\begin{verbatim}
## Observations: 31
## Variables: 3
## $ Girth  <dbl> 8.3, 8.6, 8.8, 10.5, 10.7, 10.8, 11.0, 11.0, 11.1, 11.2...
## $ Height <dbl> 70, 65, 63, 72, 81, 83, 66, 75, 80, 75, 79, 76, 76, 69,...
## $ Volume <dbl> 10.3, 10.3, 10.2, 16.4, 18.8, 19.7, 15.6, 18.2, 22.6, 1...
\end{verbatim}

Note that these two functions do essentially the same thing. In some
cases one is better than the other but that tends to not be clear until
you try it. As a rule of thumb, if you have a lot of colummns in your
data, you probably want to be using glimpse since it shows all of the
column names where head might overflow onto several pages.

Now let's do some manipulations to our data.

\begin{Shaded}
\begin{Highlighting}[]
\NormalTok{trees <-}\StringTok{ }\KeywordTok{tbl_df}\NormalTok{(trees) }\CommentTok{# coerce dataframe into a tbl }

\CommentTok{# find the min and max for the girth }
\NormalTok{dplyr}\OperatorTok{::}\KeywordTok{summarise}\NormalTok{(trees, }\DataTypeTok{minimum =} \KeywordTok{min}\NormalTok{(Girth))}
\end{Highlighting}
\end{Shaded}

\begin{verbatim}
## # A tibble: 1 x 1
##   minimum
##     <dbl>
## 1     8.3
\end{verbatim}

\begin{Shaded}
\begin{Highlighting}[]
\NormalTok{dplyr}\OperatorTok{::}\KeywordTok{summarise_each}\NormalTok{(trees, }\KeywordTok{funs}\NormalTok{(min))}
\end{Highlighting}
\end{Shaded}

\begin{verbatim}
## Warning: funs() is soft deprecated as of dplyr 0.8.0
## Please use a list of either functions or lambdas: 
## 
##   # Simple named list: 
##   list(mean = mean, median = median)
## 
##   # Auto named with `tibble::lst()`: 
##   tibble::lst(mean, median)
## 
##   # Using lambdas
##   list(~ mean(., trim = .2), ~ median(., na.rm = TRUE))
## This warning is displayed once per session.
\end{verbatim}

\begin{verbatim}
## # A tibble: 1 x 3
##   Girth Height Volume
##   <dbl>  <dbl>  <dbl>
## 1   8.3     63   10.2
\end{verbatim}

\begin{Shaded}
\begin{Highlighting}[]
\CommentTok{# many other function available}
\NormalTok{dplyr}\OperatorTok{::}\KeywordTok{summarise_each}\NormalTok{(trees, }\KeywordTok{funs}\NormalTok{(IQR))}
\end{Highlighting}
\end{Shaded}

\begin{verbatim}
## # A tibble: 1 x 3
##   Girth Height Volume
##   <dbl>  <dbl>  <dbl>
## 1  4.20      8   17.9
\end{verbatim}

\textbf{Exercise 1:} find the maximum of each column in the trees
dataset.

One of the nice things about these packages, is that they come with a
cheatsheet which is very helpful to have nearby. You can find them via
Google.
\href{https://rstudio.com/wp-content/uploads/2015/02/data-wrangling-cheatsheet.pdf}{Link
for dplyr cheatsheet}. Take a look at this cheatsheet and try some of
the functions.

The three that I find especially useful are select, filter, and mutate.
While select and filter are self-explanatory, mutate is not. The mutate
function in dplyr performs some operation and appends a colummn to the
end of your data table. Let's see some examples.

\begin{Shaded}
\begin{Highlighting}[]
\CommentTok{# take note of the piping operator}

\CommentTok{# pipes pass the output of the commands preceding it into the first }
\CommentTok{# argument of the next command }

\NormalTok{trees }\OperatorTok\StringTok{ }\NormalTok{dplyr}\OperatorTok{::}\KeywordTok{mutate}\NormalTok{(}\DataTypeTok{Girth_cm =}\NormalTok{ Girth}\OperatorTok{*}\FloatTok{2.23}\NormalTok{)}
\end{Highlighting}
\end{Shaded}

\begin{verbatim}
## # A tibble: 31 x 4
##    Girth Height Volume Girth_cm
##    <dbl>  <dbl>  <dbl>    <dbl>
##  1   8.3     70   10.3     18.5
##  2   8.6     65   10.3     19.2
##  3   8.8     63   10.2     19.6
##  4  10.5     72   16.4     23.4
##  5  10.7     81   18.8     23.9
##  6  10.8     83   19.7     24.1
##  7  11       66   15.6     24.5
##  8  11       75   18.2     24.5
##  9  11.1     80   22.6     24.8
## 10  11.2     75   19.9     25.0
## # ... with 21 more rows
\end{verbatim}

\begin{Shaded}
\begin{Highlighting}[]
\NormalTok{trees }\OperatorTok\StringTok{ }
\StringTok{  }\NormalTok{dplyr}\OperatorTok{::}\KeywordTok{mutate}\NormalTok{(}\DataTypeTok{Girth_cm =} \KeywordTok{round}\NormalTok{(Girth}\OperatorTok{*}\FloatTok{2.23}\NormalTok{,}\DecValTok{2}\NormalTok{)) }\OperatorTok\StringTok{ }
\StringTok{  }\NormalTok{dplyr}\OperatorTok{::}\KeywordTok{select}\NormalTok{(Height, Volume, Girth_cm)}
\end{Highlighting}
\end{Shaded}

\begin{verbatim}
## # A tibble: 31 x 3
##    Height Volume Girth_cm
##     <dbl>  <dbl>    <dbl>
##  1     70   10.3     18.5
##  2     65   10.3     19.2
##  3     63   10.2     19.6
##  4     72   16.4     23.4
##  5     81   18.8     23.9
##  6     83   19.7     24.1
##  7     66   15.6     24.5
##  8     75   18.2     24.5
##  9     80   22.6     24.8
## 10     75   19.9     25.0
## # ... with 21 more rows
\end{verbatim}

\begin{Shaded}
\begin{Highlighting}[]
\NormalTok{trees }\OperatorTok\StringTok{ }
\StringTok{  }\NormalTok{dplyr}\OperatorTok{::}\KeywordTok{mutate}\NormalTok{(}\DataTypeTok{Girth_cm =} \KeywordTok{round}\NormalTok{(Girth}\OperatorTok{*}\FloatTok{2.23}\NormalTok{,}\DecValTok{2}\NormalTok{)) }\OperatorTok\StringTok{ }
\StringTok{  }\NormalTok{dplyr}\OperatorTok{::}\KeywordTok{select}\NormalTok{(Height, Volume, Girth_cm) }\OperatorTok\StringTok{ }
\StringTok{  }\NormalTok{dplyr}\OperatorTok{::}\KeywordTok{filter}\NormalTok{(Girth_cm }\OperatorTok{>}\StringTok{ }\DecValTok{33}\NormalTok{)}
\end{Highlighting}
\end{Shaded}

\begin{verbatim}
## # A tibble: 8 x 3
##   Height Volume Girth_cm
##    <dbl>  <dbl>    <dbl>
## 1     72   38.3     35.7
## 2     77   42.6     36.4
## 3     81   55.4     38.6
## 4     82   55.7     39.0
## 5     80   58.3     39.9
## 6     80   51.5     40.1
## 7     80   51       40.1
## 8     87   77       45.9
\end{verbatim}

Note thatthe results displayed above are NOT stored into memory but
merely displayed. If we wanted to store them and use them later, we need
to use the assignment operator and store it to a variable in memory.

\begin{Shaded}
\begin{Highlighting}[]
\NormalTok{selected_trees <-}\StringTok{ }\NormalTok{trees }\OperatorTok\StringTok{ }
\StringTok{  }\NormalTok{dplyr}\OperatorTok{::}\KeywordTok{mutate}\NormalTok{(}\DataTypeTok{Girth_cm =} \KeywordTok{round}\NormalTok{(Girth}\OperatorTok{*}\FloatTok{2.23}\NormalTok{,}\DecValTok{2}\NormalTok{)) }\OperatorTok\StringTok{ }
\StringTok{  }\NormalTok{dplyr}\OperatorTok{::}\KeywordTok{select}\NormalTok{(Height, Volume, Girth_cm) }\OperatorTok\StringTok{ }
\StringTok{  }\NormalTok{dplyr}\OperatorTok{::}\KeywordTok{filter}\NormalTok{(Girth_cm }\OperatorTok{>}\StringTok{ }\DecValTok{33}\NormalTok{)}

\NormalTok{selected_trees }\CommentTok{# same as above }
\end{Highlighting}
\end{Shaded}

\begin{verbatim}
## # A tibble: 8 x 3
##   Height Volume Girth_cm
##    <dbl>  <dbl>    <dbl>
## 1     72   38.3     35.7
## 2     77   42.6     36.4
## 3     81   55.4     38.6
## 4     82   55.7     39.0
## 5     80   58.3     39.9
## 6     80   51.5     40.1
## 7     80   51       40.1
## 8     87   77       45.9
\end{verbatim}

\textbf{Exercise 2:} Using the dplyr cheatsheet, create a data table
containing only the trees which have a girth greater than 20cm and less
than 36cm. Arrange them in ascending order based on the Height.

\begin{verbatim}
## # A tibble: 21 x 3
##    Height Volume Girth_cm
##     <dbl>  <dbl>    <dbl>
##  1     64   24.9     30.8
##  2     66   15.6     24.5
##  3     69   21.3     26.1
##  4     71   25.7     30.6
##  5     72   16.4     23.4
##  6     72   38.3     35.7
##  7     74   22.2     28.8
##  8     74   36.3     32.3
##  9     75   18.2     24.5
## 10     75   19.9     25.0
## # ... with 11 more rows
\end{verbatim}

\hypertarget{plotting}{%
\subsubsection{Plotting}\label{plotting}}

Now that we have cleaned our data to our subjects of interest we can
start plotting this data to get some more insight into what exactly is
happening. We will be using ggplot2, a layer based graphical tool based
on the Grammar of Graphics guidelines. The way it works is by adding
data, then mapping variables to aesthetic elements of the plot like the
``x-axis'' or the ``colour''; then you tell it what to draw (points,
lines, bars, etc); then you annotate the plot with axis labels and other
things (this description is credited to
\href{https://awstringer1.github.io/ssu-r-workshop/ssu-r-workshop.html}{Alex
Stringer's} R workshop). So let's do exactly that. We will plot this
stepwise.

\begin{Shaded}
\begin{Highlighting}[]
\NormalTok{plot1 <-}\StringTok{ }\NormalTok{solution }\OperatorTok\StringTok{ }\KeywordTok{ggplot}\NormalTok{(}\KeywordTok{aes}\NormalTok{(}\DataTypeTok{x =}\NormalTok{ Height))}
\NormalTok{plot1}
\end{Highlighting}
\end{Shaded}

\includegraphics{R_workshop_files/figure-latex/unnamed-chunk-1-1.pdf}

\begin{Shaded}
\begin{Highlighting}[]
\CommentTok{# add my favourite theme }
\NormalTok{plot1 <-}\StringTok{ }\NormalTok{plot1 }\OperatorTok{+}\StringTok{ }\KeywordTok{theme_minimal}\NormalTok{()}
\NormalTok{plot1 }\CommentTok{# much better :D }
\end{Highlighting}
\end{Shaded}

\includegraphics{R_workshop_files/figure-latex/unnamed-chunk-1-2.pdf}

Now let's start making some plots. The first one we will look at is a
bar graph.

\begin{Shaded}
\begin{Highlighting}[]
\NormalTok{plot1 <-}\StringTok{ }\NormalTok{plot1 }\OperatorTok{+}\StringTok{ }\KeywordTok{geom_histogram}\NormalTok{(}\DataTypeTok{binwidth =} \DecValTok{3}\NormalTok{, }\DataTypeTok{fill =} \StringTok{"gold"}\NormalTok{, }\DataTypeTok{colour =} \StringTok{"navy"}\NormalTok{, }\DataTypeTok{alpha =} \FloatTok{0.7}\NormalTok{)}
\NormalTok{plot1}
\end{Highlighting}
\end{Shaded}

\includegraphics{R_workshop_files/figure-latex/unnamed-chunk-2-1.pdf}

Let's add some labels to our graph.

\begin{Shaded}
\begin{Highlighting}[]
\NormalTok{plot1 <-}\StringTok{ }\NormalTok{plot1 }\OperatorTok{+}\StringTok{ }\KeywordTok{labs}\NormalTok{(}\DataTypeTok{title =} \StringTok{"Histogram of Selected Tree Heights"}\NormalTok{,}
  \DataTypeTok{xlab =} \StringTok{"Height (inches)"}\NormalTok{)}
\NormalTok{plot1}
\end{Highlighting}
\end{Shaded}

\includegraphics{R_workshop_files/figure-latex/unnamed-chunk-3-1.pdf}

The way we've been doing it has been by adding everything layer by
layer. This can be a bit tedious and hence can be avoided by taking a
one shot approach. The full code for the histogram above is:

\begin{Shaded}
\begin{Highlighting}[]
\NormalTok{plot1 <-}\StringTok{ }\NormalTok{solution }\OperatorTok\StringTok{ }\KeywordTok{ggplot}\NormalTok{(}\KeywordTok{aes}\NormalTok{(}\DataTypeTok{x =}\NormalTok{ Height)) }\OperatorTok{+}\StringTok{ }
\StringTok{  }\KeywordTok{geom_histogram}\NormalTok{(}\DataTypeTok{binwidth =} \DecValTok{3}\NormalTok{, }\DataTypeTok{fill =} \StringTok{"gold"}\NormalTok{, }\DataTypeTok{colour =} \StringTok{"navy"}\NormalTok{, }\DataTypeTok{alpha =} \FloatTok{0.7}\NormalTok{) }\OperatorTok{+}\StringTok{ }
\StringTok{  }\KeywordTok{labs}\NormalTok{(}\DataTypeTok{title =} \StringTok{"Histogram of Selected Tree Heights"}\NormalTok{,}
  \DataTypeTok{xlab =} \StringTok{"Height (inches)"}\NormalTok{) }\OperatorTok{+}\StringTok{ }
\StringTok{  }\KeywordTok{theme_minimal}\NormalTok{()}

\NormalTok{plot1}
\end{Highlighting}
\end{Shaded}

We now turn our attention to line graphs. In this case, we will take the
one-shot approach to producing this graph.

\begin{Shaded}
\begin{Highlighting}[]
\NormalTok{plot2 <-}\StringTok{ }\NormalTok{solution }\OperatorTok\StringTok{ }
\StringTok{  }\KeywordTok{ggplot}\NormalTok{(}\KeywordTok{aes}\NormalTok{(}\DataTypeTok{x =}\NormalTok{ Height, }\DataTypeTok{y =}\NormalTok{ Volume)) }\OperatorTok{+}
\StringTok{  }\KeywordTok{geom_point}\NormalTok{(}\DataTypeTok{colour =} \StringTok{"navy"}\NormalTok{, }\DataTypeTok{alpha =} \FloatTok{0.7}\NormalTok{) }\OperatorTok{+}\StringTok{ }
\StringTok{  }\KeywordTok{labs}\NormalTok{(}\DataTypeTok{title =} \StringTok{"Volume of CO2 vs. Height of Trees"}\NormalTok{,}
       \DataTypeTok{x =} \StringTok{"Height (inches)"}\NormalTok{,}
       \DataTypeTok{y =} \StringTok{"Volume of CO2 (m^3)"}\NormalTok{) }\OperatorTok{+}
\StringTok{  }\KeywordTok{geom_smooth}\NormalTok{(}\DataTypeTok{method=}\StringTok{'lm'}\NormalTok{, }\DataTypeTok{colour =} \StringTok{"gold"}\NormalTok{)}\OperatorTok{+}
\StringTok{  }\KeywordTok{theme_minimal}\NormalTok{()}

\NormalTok{plot2 <-}\StringTok{ }\NormalTok{solution }\OperatorTok\StringTok{ }
\StringTok{  }\KeywordTok{ggplot}\NormalTok{(}\KeywordTok{aes}\NormalTok{(}\DataTypeTok{x =}\NormalTok{ Height, }\DataTypeTok{y =}\NormalTok{ Volume)) }\OperatorTok{+}
\StringTok{  }\KeywordTok{geom_jitter}\NormalTok{(}\DataTypeTok{colour =} \StringTok{"navy"}\NormalTok{, }\DataTypeTok{alpha =} \FloatTok{0.7}\NormalTok{) }\OperatorTok{+}\StringTok{ }
\StringTok{  }\KeywordTok{labs}\NormalTok{(}\DataTypeTok{title =} \StringTok{"Volume of CO2 vs. Height of Trees"}\NormalTok{,}
       \DataTypeTok{x =} \StringTok{"Height (inches)"}\NormalTok{,}
       \DataTypeTok{y =} \StringTok{"Volume of CO2 (m^3)"}\NormalTok{) }\OperatorTok{+}
\StringTok{  }\KeywordTok{geom_smooth}\NormalTok{(}\DataTypeTok{method=}\StringTok{'lm'}\NormalTok{, }\DataTypeTok{colour =} \StringTok{"gold"}\NormalTok{)}\OperatorTok{+}
\StringTok{  }\KeywordTok{theme_minimal}\NormalTok{()}

\NormalTok{plot2}
\end{Highlighting}
\end{Shaded}

\includegraphics{R_workshop_files/figure-latex/Line Graph-1.pdf}


\end{document}
